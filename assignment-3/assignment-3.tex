% Created 2026-01-23 Fri 22:12
% Intended LaTeX compiler: pdflatex
\documentclass[hidelinks, 11pt]{article}
\usepackage[utf8]{inputenc}
\usepackage[T1]{fontenc}
\usepackage{graphicx}
\usepackage{longtable}
\usepackage{wrapfig}
\usepackage{rotating}
\usepackage[normalem]{ulem}
\usepackage{amsmath}
\usepackage{amssymb}
\usepackage{capt-of}
\usepackage{hyperref}
\makeatletter \@ifpackageloaded{geometry}{\geometry{margin=2cm}}{\usepackage[margin=2cm]{geometry}} \makeatother
\def \homeworkNumber{3}
\usepackage{siunitx}
\usepackage{amsmath}
\usepackage{amssymb}
\usepackage{mathtools}
\usepackage{circuitikz}
\usepackage{float}
\usepackage{karnaugh-map}
\usepackage[sfdefault]{plex-sans}
\usepackage{isomath}
\usepackage[italic]{mathastext}
\usepackage[scale=0.90]{plex-mono}
\usepackage{fancyhdr}
\pagestyle{fancyplain}
\chead{Assignment \homeworkNumber}
\lhead{Bach \textsc{Wongwandanee}}
\rhead{ECE 403}
\author{Waridh (Bach) Wongwandanee}
\date{\today}
\title{ECE 403 Assignment 2}
\hypersetup{
 pdfauthor={Waridh (Bach) Wongwandanee},
 pdftitle={ECE 403 Assignment 2},
 pdfkeywords={},
 pdfsubject={},
 pdfcreator={Emacs 30.2 (Org mode 9.7.39)}, 
 pdflang={English}}
\usepackage{biblatex}

\begin{document}

\section*{Question 1}
\label{sec:org50bea63}
Produce a schematic design of 2-to-1 non-restoring multiplexer gate using CMOS transmission gates plus other static CMOS gates. Use inputs \texttt{\{s, d0, d1\}} and output \texttt{y}. Inverted signals are not provided.

\begin{table}[htbp]
\caption{The truth table of the 2-to-1 non-restoring multiplexer.}
\centering
\begin{tabular}{rllr}
\texttt{s} & \texttt{d1} & \texttt{d0} & y\\
\hline
0 & X & 0 & 0\\
0 & X & 1 & 1\\
1 & 0 & X & 0\\
1 & 1 & X & 1\\
\end{tabular}
\end{table}


\begin{figure}[htbp]
\centering
\includegraphics[width=0.4\linewidth]{images/q1-schematic.png}
\caption{The schematic for the 2-to-1 nonrestoring multiplexer gate.}
\end{figure}
\section*{Question 2}
\label{sec:org122cd61}
Draw a sticks diagram of this mux.

Optimize and count the number of:
\begin{enumerate}
\item transistors
\item diffusion regions
\item metal-diffusion contacts
\end{enumerate}

The layout of the 2-to-1 non-restoring multiplexer is available in figure \ref{fig-q2-layout}. The count of the components used in the design is available in table \ref{tab-q2-count}.

\begin{figure}[htbp]
\centering
\includegraphics[width=0.7\linewidth]{images/q2-layout.png}
\caption{\label{fig-q2-layout}The layout for the 2-to-1 non-restoring multiplexer where inverted signals are not provided.}
\end{figure}

\begin{table}[htbp]
\caption{\label{tab-q2-count}The number of components used in the implementation of the non-restoring 2-to-1 multiplexer.}
\centering
\begin{tabular}{lr}
component & number\\
\hline
transistors & 6\\
diffusion regions & 4\\
metal contacts & 13\\
\end{tabular}
\end{table}
\section*{Question 3}
\label{sec:orgf6c445e}
Describe how the delay grows with the number of stages \((n)\) of
a circuit with these multiplexers connected together (one output
to the next input) in a combinational path. Remember that
each transistor has resistance and capacitance.
\end{document}
